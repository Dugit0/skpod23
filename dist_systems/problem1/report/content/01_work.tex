\section*{Задача}
Реализовать программу, моделирующую выполнение протокола голосования для
12 файловых серверов при помощи пересылок MPI типа точка-точка. Получить
временную оценку времени выполнения одним процессом 3-х операций записи и
10 операций чтения $N$ байтов информации с файлом, расположенным (размноженным)
на 12 серверах. Определить оптимальные значения кворума чтения и кворума записи
для $N=300$. Время старта равно 100, время передачи байта равно 1
($Ts=100$,$Tb=1$).


\section*{Реализация}

Реализуем алгорим работы файловых серверов при помощи пересылок MPI типа
точка-точка. Это потребует небольшой модификации алгоритма, описанного на
лекциях и в книге Э.С.\,Таненбаума "Распределенные системы"\,, для корректной
обработке дедлоков без назначения тайм-аута, используя только передачи типа
точка-точка.

На рис. \ref{fig:all} приведена подробная блок-схема модифицированного
алгоритма. Подпрограмма "Получить все ответы"\, приведена на рис.
\ref{fig:get_all_req}.

Части схемы в большем масштабе приведены на рис. \ref{fig:read},
\ref{fig:write}, \ref{fig:idle}.

\begin{figure}[H]
    \centering
    \includegraphics[width=1.\linewidth,center]{all.png}
    \caption{Схема работы алгоритма}
    \label{fig:all}
\end{figure}

\begin{figure}[H]
    \centering
    \includegraphics[width=.4\linewidth,center]{get_all_req.png}
    \caption{Подпрограмма "Получить все ответы"\,}
    \label{fig:get_all_req}
\end{figure}

Части схемы с рис. \ref{fig:all} в большем масштабе:

\begin{figure}[H]
    \centering
    \includegraphics[width=.7\linewidth,center]{read.png}
    \caption{Схема чтения}
    \label{fig:read}
\end{figure}

\begin{figure}[H]
    \centering
    \includegraphics[width=.6\linewidth,center]{write.png}
    \caption{Схема записи}
    \label{fig:write}
\end{figure}

\begin{figure}[H]
    \centering
    \includegraphics[width=.7\linewidth,center]{idle.png}
    \caption{Схема простаивания}
    \label{fig:idle}
\end{figure}


\section*{Теоретическая оценка алгоритма}

\subsection*{Условие:}

Получить временную оценку времени выполнения одним процессом $W = 3$ операций
записи и $R = 10$ операций чтения $M$ байтов информации с файлом, расположенным
(размноженным) на $N = 12$ серверах. Определить оптимальные значения кворума чтения
и кворума записи для $M = 300$. Время старта равно $T_s = 100$, время передачи
байта равно $T_b = 1$.

\subsection*{Решение:}

\begin{itemize}
    \item Обозначим время чтения 1 байта информации, как $T_r$, а время
        записи -- $T_w$.
    \item Обозначим кворум чтения, как $N_r$, а кворум записи -- $N_w$.
        Естественно, что $N_r + N_w > N$ и $N_w > N / 2$.
    \item Будем считать, что для запроса на чтение/запись и отправки
        согласия/отказа в ответ на такой запрос хватает 1 байта. И такие
        запросы сервер посылает и принимает последовательно,
        т.к. тип пересылок точка-точка.
\end{itemize}

1. \textit{Оценка снизу.}

Пусть $W$ операции записи и $R$ операций чтения
происходят на одном сервере и являются единственными операциями, которые
происходят за рассматриваемый промежуток времени. Таким образом передача
файла текущему серверу происходить не будет, т.к. текущий сервер всегда имеет
самую актуальную версию.

Оценим операцию чтения, как время необходимое на запрос на $N_r$ серверов,
ответ от них и отправку подтверждения, что они снова могут голосовать.
$$
3*(N_r * 1 * T_b) = 3N_rT_b
$$

Оценим операцию записи, как время необходимое на запрос на $N_w$ серверов,
ответ от них, отправку подтверждения, что они снова могут голосовать и новой
версии файла.
$$
3*(N_w * 1 * T_b) + N_w*M*T_b = N_wT_b(3 + M)
$$

Итоговая оценка снизу:
$$
T_s + 3N_rT_bR + N_wT_b(3 + M)W = 100 + 30N_r + 909N_w
$$

2. \textit{Оценка сверху.}

Пусть $W$ операции записи и $R$ операций чтения происходят на одном сервере, но
каждый раз между ними происходит запись в файл и обновление его версии,
инициированное сторонним сервером.

Для простоты будем считать, что сторонний сервер не обращается к нашему
серверу, т.к. тогда расходы на передачу версии файла могут неограниченно расти,
если предположить, что таких записей от сторонних серверов может быть
бесконечно много.

Оценим операцию чтения, как время необходимое на запрос на $N_r$ серверов,
ответ от них, запрос на получение самой новой версии файла, получение файла и
отправку подтверждения серверам, что они снова могут голосовать.
$$
N_r*1*T_b + N_r*1*T_b + 1*T_b + M*T_b + N_r*1*T_b = (3N_r + 1 + M)T_b
$$

Оценим операцию записи, как время необходимое на запрос на $N_w$ серверов,
ответ от них, запрос на получение самой новой версии файла, получение файла и
отправку подтверждения, что они снова могут голосовать и новой версии файла.
$$
N_w*1*T_b + N_w*1*T_b + 1*T_b + M*T_b + N_w*1*T_b + N_w*M*T_b = (3N_w + N_wM + 1 + M)T_b
$$

Итоговая оценка сверху:
$$
T_s + (3N_r + 1 + M)T_bR + (3N_w + N_wM + 1 + M)T_bW = 4013 + 30N_r + 909N_w
$$

3. \textit{Определение оптимального значения кворумов чтения и записи.}

В п.1 и п.2 была получена оценка времени выполнения указанных операций чтения и
записи:
$$
100 + 30N_r + 909N_w \le T \le 4013 + 30N_r + 909N_w
$$

Таким образом имеем

$$
\begin{cases}
    \min (30N_r + 909N_w)\\
    N_r + N_w > N\\
    N_w > N / 2
\end{cases}
$$

Решая эту систему в целых числах, получаем ответ $N_r = 6$, $N_w = 7$.

\textbf{Ответ:} $N_r = 6$, $N_w = 7$.
