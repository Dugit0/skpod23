\section{Выводы}
Очевидно, что RedBlack3D алгоритм оказался гораздо быстрее, однако это компенсируется тем, что данный алгоритм в принципе не гарантирует какую-либо точность, а лишь позволяет обеспечить сходимость узкого класса методов.

Можно заметить, что с определенного порога при увеличении числа нитей алгоритм распараллеливания по гиперплоскостям дает худший результат. Скорее всего данный эффект обусловлен наличием в программе разделяемых ресурсов. Таким образом большое число нитей увеличивает расходы на передачу управления между ними, что замедляет работу программы.

