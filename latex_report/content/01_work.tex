\section{Сделанные модификации}


\subsection{Оптимизации без использования OpenMP}
Порядок переменных в циклах был заменен с \code{k}, \code{j}, \code{i} на \code{i}, \code{j}, \code{k}. Это ускорило работу программы на 25\% за счет оптимизации доступа к памяти.

\subsection{Оптимизации с использованием OpenMP}
Т.к. функции \code{void init(void)} и \code{void verify(void)} не имеют регулярной зависимости по данным, то в них были добавлены строки
\begin{lstlisting}
#pragma omp parallel for collapse(3) default(none) private(i, j, k) shared(A)
\end{lstlisting}
и
\begin{lstlisting}
#pragma omp parallel for collapse(3) default(none) private(i, j, k) shared(A) reduction(+:s)
\end{lstlisting}
соответственно.

\subsection{Алгоритм RedBlack3D}
Модифицируем алгоритм RedBlack2D для трехмерного случая, высчитывая четность суммы индексов \code{(i + j + k) \% 2} и разбивая вычисления на 2 подцикла. Подробнее реализацию можно посмотреть в файле \code{openmp\_redblack.c}.

\subsection{Распараллеливание по гиперплоскостям куба}
Изначально данной задачи была написана функция, вычисляющая индексы ячеек, принадлежащих каждой из гиперплоскостей. Данная функция хорошо распараллеливается, однако из-за общей структуры данных она имеет критическую секцию, что замедляет ее работу. Из-за этого данный код пришлость полностью переписать.

Был использован другой подход к реализации. Можно заметить, что вычисление индексов $H$-ой гиперплоскости сводится к классической комбинаторной задаче о шариках и перегородках.

Пусть есть $N$ шариков и $M$ перегородок. Перегородки ставятся в промежудках между шариками и образуют "ящики". Рассмотрим пример для $N=3$ и $M=2$.
\begin{center}
    \begin{tabular}{ c | c }
        Схема & Индексы\\
        \hline
        $ \mathtt{|| \circ  \circ  \circ} $ & $(0, 0, 3)$ \\
        $ | \circ | \circ  \circ $ & $(0, 1, 2)$ \\
        $ | \circ  \circ | \circ $ & $(0, 2, 1)$ \\
        $ | \circ  \circ  \circ | $ & $(0, 3, 0)$ \\
        $ \circ || \circ  \circ $ & $(1, 0, 2)$ \\
        $ \circ | \circ | \circ $ & $(1, 1, 1)$ \\
        $ \circ | \circ  \circ | $ & $(1, 2, 0)$ \\
        $ \circ  \circ || \circ $ & $(2, 0, 1)$ \\
        $ \circ  \circ | \circ | $ & $(2, 1, 0)$ \\
        $ \circ  \circ  \circ || $ & $(3, 0, 0)$ \\
    \end{tabular}
\end{center}

Легко заметить, что данные индексы образуют 3-ю гиперплоскость в трехмерном массиве. Используем этот факт в реализации задачи. Перепишем функцию relax так, чтобы обход шел не по индексам, а по элементам гиперплоскости. Подробнее реализацию можно посмотреть в файле \code{openmp\_hyperplain.c}.




