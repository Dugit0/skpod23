\section{Сделанные модификации}


\subsection{Оптимизации без использования OpenMP}
Порядок переменных в циклах был заменен с \code{k}, \code{j}, \code{i} на \code{i}, \code{j}, \code{k}. Это ускорило работу программы на 25\% за счет оптимизации доступа к памяти.

\subsection{Оптимизации с использованием OpenMP}
Т.к. функции \code{void init(void)} и \code{void verify(void)} не имеют регулярной зависимости по данным, то в них были добавлены строки
\begin{lstlisting}
#pragma omp parallel for collapse(3) default(none) private(i, j, k) shared(A)
\end{lstlisting}
и
\begin{lstlisting}
#pragma omp parallel for collapse(3) default(none) private(i, j, k) shared(A) reduction(+:s)
\end{lstlisting}
соответственно.

\subsection{Алгоритм RedBlack3D}
Модифицируем алгоритм RedBlack2D для трехмерного случая, высчитывая четность суммы индексов \code{(i + j + k) \% 2} и разбивая вычисления на 2 подцикла. Подробнее реализацию можно посмотреть в файле \code{openmp\_redblack.c}.

\subsection{Распараллеливание по гиперплоскостям куба}
Для данной задачи была написана функция, вычисляющая индексы ячеек, принадлежащих каждой из гиперплоскостей. Данная функция хорошо распараллеливается, однако из-за общей структуры данных она имеет критическую секцию, что замедляет ее работу. Также была переработана функция relax так, чтобы обход шел не по индексам, а по элементам гиперплоскости. Подробнее реализацию можно посмотреть в файле \code{openmp\_hyperplain.c}.




